\documentclass[11pt,letterpaper]{article}
\usepackage[spanish]{babel}
%\usepackage[ansinew]{inputenc}
\usepackage[utf8]{inputenc}
%\usepackage[latin1]{inputenc}
\usepackage[letterpaper,includeheadfoot, top=0.5cm, bottom=3.0cm, right=2.0cm, left=2.0cm]{geometry}
\renewcommand{\familydefault}{\sfdefault}

\usepackage{graphicx}
\usepackage{color}
\usepackage{hyperref}
\usepackage{amssymb}
\usepackage{url}
%\usepackage{pdfpages}
\usepackage{fancyhdr}
\usepackage{hyperref}
\usepackage{subfig}

\usepackage{listings} %Codigo
\lstset{language=C, tabsize=4,framexleftmargin=5mm,breaklines=true}

\begin{document}
% --------------- ---------PORTADA --------------------------------------------
\newpage
\pagestyle{fancy}
\fancyhf{}
%-------------------- CABECERA ---------------------
\fancyhead[L]{ \includegraphics[scale=0.9]{img/logo_die.pdf} }
%------------------ TÍTULO -----------------------
\vspace*{6cm}
\begin{center}
\Huge  {Informe Preliminar} \\
\vspace{1cm}
\LARGE {EL6908 - Introducción al Trabajo de Título}\\
\vspace{0.5cm}
\LARGE {\textit{Diseño e Implementación de una plataforma para experimentos en microgravedad y de electrónica en ambiente hostil con Nano-Sat\'elites}}\\
%\vspace{1cm}
%\small {Título pequeño} 
\end{center}
%----------------- NOMBRES ------------------------
\vfill
\begin{flushright}
\begin{tabular}{ll}
\textbf{Autor} &: Jos\'e Ogalde O.\\
\textbf{Profesor Guía} &: Marcos Díaz Q.\\
\textbf{Profesor EL6908} &: Jorge Lopez H.\\
& \today\\
& Santiago, Chile.
\end{tabular}
\end{flushright}

% ·············· ENCABEZADO ············
\newpage
\pagestyle{fancy}
\fancyhf{}
%\fancyhead[L]{\rightmark}
\fancyhead[L]{\small \rm \textit{Sección \rightmark}}
\fancyhead[R]{\small \rm \textbf{\thepage}}
\fancyfoot[L]{\small \rm \textit{Informe Final}}
\fancyfoot[R]{\small \rm \textit{EL6908 - Introducción al Trabajo de Título}}
%\fancyfoot[C]{\thepage}
\renewcommand{\sectionmark}[1]{\markright{\thesection.\ #1}}
\renewcommand{\headrulewidth}{0.5pt}
\renewcommand{\footrulewidth}{0.5pt}

% =============== INDICE ===============

\tableofcontents
\listoffigures
% \listoftables

% =============== IDENTIFICAION ===============
\newpage
\section{Identificación}

\subsection{Alumno}

\begin{itemize}
	\item \textbf{Nombre:} Jos\'e Alberto Ogalde Ortiz
	\item \textbf{Rut:} 18.011.135-2
	\item \textbf{Nº de matrícula:} 2010115315
	\item \textbf{Dirección:} Avenida Blanco Encalada 1723, Dpto 102, Santiago.
	\item \textbf{Tel\'efono:} +56 9 6685 4005
	\item \textbf{Correo:} jose.ogalde@ing.uchile.cl
\end{itemize}

\subsection{Profesor Guía}

\begin{itemize}
	\item \textbf{Nombre:} Marcos Díaz Quezada
	\item \textbf{Dirección:} Av. Tupper 2007, Of. 510, Santiago, Chile
	\item \textbf{Teléfono:} Fono: +56 9 9051 0540
	\item \textbf{Correo:} mdiazq@ing.uchile.cl
\end{itemize}

\subsection{Profesor Integrante}

\begin{itemize}
	\item \textbf{Nombre:} Claudio Falcón
	\item \textbf{Dirección:} Av. Blanco  Encalada 2008, Santiago, Chile.
	\item \textbf{Teléfono:} Fono: +56 2 9788 4695
	\item \textbf{Correo:} cfalcon@ing.uchile.cl
\end{itemize}

% =============== CUERPO ===============
\newpage
\section{Definición del Tema de Trabajo de Título}

\subsection{Título del tema}

Diseño e Implementación de una plataforma para experimentos en microgravedad y de electrónica en ambiente hostil con Nano-Sat\'elites

\subsection{Fundamentación y objetivos generales}

Esta memoria se enmarca en el desarrollo del proyecto SUCHAI que consiste en la implementación, lanzamiento y operación de un nano-satélite Cubesat, siendo esta la primera aproximación en esta materia para la universidad y el país. La misión de un sat\'elite es lo que define el objetivo y propósito de su lanzamiento. En el caso de los nano-sat\'elites, la misión está dominada por la cantidad y tipos de \textit{Payloads} que llevan consigo. Estos \textit{payloads} normalmente corresponden a plataformas en las cuales se realiza alguna actividad de inter\'es para la misión. Por ejemplo, en el proyecto SUCHAI los payloads a bordo del sat\'elite corresponden a plataformas electrónicas que ejecutan experimentos físicos/geofísicos, pues uno de los objetivos de la misión de SUCHAI es el estudio científico sobre los fenómenos que gobiernan dichos experimentos.

El objetivo de este trabajo es el diseño, desarrollo e implementación del una plataforma de tipo \textit{payload} que permita ejecutar experimentos físicos y de electrónica a bordo de un nano-sat\'elite. Esto se puede divisir en tres objetivos generales:
\begin{itemize}
\item Diseñar e implementar una plataforma electrónica de tipo payload que permita ejecutar experimentos físicos y/o de electrónica dentro de un nano-sat\'elite.
%\item Diseñar y contruir una PCB que sea capaz de almacenar el sistema bajo observación junto a los sensores, actuadores y controladores necesarios para tomar las medidas adecuadas.
%\item Diseñar e implementar una interfaz de sotfware que permita manipular de forma sencilla la ejecución del experimento.
\end{itemize}

\subsection{Bibliografía y Estado del Arte}

Se considera la siguiente bibliografía para el desarrollo de este trabajo de título:

\begin{itemize}
	\item L. Di Jasio \textit{Programming 16-Bit PIC Microcontrollers in C} , 1st ed. USA: Elsevier, 2007.
	\item Programming 16-Bit PIC Microcontrollers in C, Second Edition: Learning to Fly the PIC 24
\end{itemize}

\subsection{Objetivos específicos}

Los objetivos específicos del proyecto se enumeran a continuación

\begin{enumerate}
 \item Diseño e implementación de una arquitectura para PCBs orientada a albergar el setup de un experimento físico y/o de electrónica.
 \item Integrar la PCB al bus del nano-sat\'elite objetivo.
 \item Implementar controladores de hardware para manipular cada uno de los elementos dentro de la PCB.
 \item Implementar una API intermedia para los controladores de hardware, de forma que permita ejecutar el experimento por medio de comandos sencillos.
 \item Agregar comandos a la API intermedia para procesamiento estadístico de los datos medidos (e.g. media, varianza, histograma, etc.).
 \item Integrar los elementos de software dentro del software de vuelo nano-sat\'elite objetivo.
 \item Realizar pruebas de los controladores de hardware para cada uno de los elementos de la plataforma.
 \item Realizar pruebas de los comandos diseñados para verificiar el buen funcionamiento del experimento.
 \item Realizar pruebas sobre el sistema integrado al nano-sat\'elite.

\end{enumerate}

El listado de objetivos presenta un orden temporal en la ejecución de estas tareas puesto que objetivo es dependiente del cumplimiento de los objetivos anteriores. El trabajo se puede considerar terminado cuando los resultados de las pruebas del sistema integrado son satisfactorias.

\subsection{Antecedentes generales}

El trabajo de memoria se enmarca en el proyecto SUCHAI (Satellite of University of Chile for Aerospace Investigation) cuyo objetivo es desarrollo, lanzamiento y operación de un nano-satélite tipo Cubesat con fines educacionales y de investigación. El proyecto es generado por el profesor Marcos Díaz quien convoca a alumnos de pregrado y postgrado a desarrollar el proyecto bajo la guía de ingenieros que han trabajo en el proyecto.

Al igual que otros nano-sat\'elites, SUCHAI alberga diferentes payloads a bordo. En particular, uno de los payloads a bordo de SUCHAI estudia la inyección de potencia a un circuito RC en ambientes hostiles como lo es el espacio. Este estudio permite analizar las diferencias (si las hay) a nivel microscópico de la física que gobierna el circuito RC. Este análisis se hace mediante relaciones estadísticas de variables macroscópicas como la potencia inyectada. El desarrollo de este trabajo de título se enmarca en la implementación del experimento reci\'en descrito.

Actualmente el proyecto se encuentra en la etapa final previa a su lanzamiento a finales de Diciembre del presente año, por lo que ya se ha trabajo bastante en la implementación del experimento. Sin embargo, se requiere un refinamiento de la plataforma actual para las futuras misiones del proyecto SUCHAI, por lo que el trabajo de título genera un aporte para los futuros experimentos físicos a bordo de las nuevas misiones de SUCHAI.

\subsection{Antecedentes específicos}

Los requerimentos y tipos de experimentos a bordo del nano-sat\'elite están en constante evolución, pero este trabajo de memoria pretende formalizar; formar una estructura base; y documentar el m\'etodo de diseño e implementación de un \textit{setup} experimental a bordo de un vehículo espacial con recursos limitados.

\subsection{Hipótesis de trabajo y metodología}
La hipot\'esis de trabajo está centrada en las restricciones impuestas por el ambiente donde se desempeña el proyecto SUCHAI, como por ejemplo:
\begin{enumerate}
\item Altos gradientes de temperatura
\item Baja o nula presión
\item Evitar conflictos al integrar la plataforma al bus del SUCHAI
\item Limitada velocidad del microcontrolador
\item Limitada memoria para almacenamiento de datos
\item Los archivos de datos enviados a tierra deben ser pequeños debido al cuellode botella presente el sistema de comunicaciones.
\end{enumerate}

\subsubsection{Plan de Trabajo}
Se considera dedicar un total de 450-525 horas de trabajo durante el semestre de EL6909 lo que se traduce en el siguiente plan de trabajo:
\begin{enumerate}
\item 6-7 horas diarias de trabajo
\item 5 días a la semana
\item 15 semanas en el semestre
\end{enumerate}

\section{Apoyo para la realización de la memoria}
\subsection{Infraestructura disponible}

\subsubsection{Instalaciones}
El proyecto se desarrolla en las dependencias de la facultad, específicamente el laboratorio SPEL (\textit{Spacial and Planetary Exploration Laboratory}), tercer piso del edificio de Electrotecnologías. Este laboratorio alberga proyectos relacionados a la exploración espacial e instrumentación para la geociencia.
\subsubsection{Software}
La mayor parte del software utilizado durante el trabajo de título corresponde al lenguaje C orientado al microcontrolador PIC24F en conjunto al software de vuelo de SUCHAI. Los detalles sobre la configuración y operación del microcontralodor se encuentran en el datasheet del mismo.

\paragraph{IDE o Entorno de desarrollo:}
Para el desarrollo de la aplicación para el microcontrolador PIC24F se utiliza el entorno de desarrollo MPLABX de Microchip, que es gratuita y se observa en la figura \ref{mplabx}; se utiliza el compilador XC16 para traducir el código del lenguaje C hacia código de máquina programable al microcontrolador.

\subsection{Condiciones Contractuales} 
El desarrollo de este trabajo se enmarca dentro del proyecto SUCHAI el cual es patrocinado y financiado por la Facultad de Ciencias Físicas y Matemáticas de la Universidad de Chile. Siendo alumno regular de la carrera Ingeniería Civil Electricista de la Universidad de Chile, parte del trabajo realizado para esta memoria ha sido considerado en el desarrollo de la Práctica Profesional II y como parte de trabajos independientes como ayudante de laboratorio de SPEL.

\subsection{Plan B}
ADSF
\end{document}
