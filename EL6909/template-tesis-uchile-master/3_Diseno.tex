\chapter{Diseño de la plataforma}


\section{Requerimientos}
\subsection{Requirimientos Operacionales}
Se refieren a las funcionalidades que se espera que la plataforma a bordo del satélite debe realizar. Son los requisitos básicos que el sistema debe cumplir para considerar
que se cuenta con una plataforma capaz de llevar a cabo la ejecución y representación del payload. La lista de requerimientos
proviene de una serie de reuniones sostenidas con los integrantes de los diferentes grupos de
trabajo, todo de acuerdo con los lineamientos del jefe de proyecto.

\subsection{Requirimientos No-Operacionales}
Por requerimientos no-operacionales, se entienden aquellos atributos asociados a la calidad de la plataforma, o bien criterios que permiten determinar cómo debería ser la interfaz que se está diseñando. A diferencia de los requerimientos operacionales que explican lo que la plataforma debería hacer, los no funcionales explican las cualidades y restricciones que guiarán el proceso de diseño.

\subsection{Requirimientos mínimos}

\section{Hardware}
\section{Software}
\subsection{Arquitectura del Software}
\subsubsection{API de comunicación}
\subsubsection{SDK para Payloads}
