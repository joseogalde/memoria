\chapter{Introducción}
\label{ch:introduction}

%\fillup{1}
%\todo{todo 1} asdfasfsdfsadafs %\missingref{me faltaría una ref!}.
%\todo[inline]{todo 2}

La etapa de diseño consiste en explicitar los requerimientos del sistema y generar una
arquitectura de software que representa de manera conceptual la solución a implementar, en
esta etapa se hace un uso extensivo del concepto de patrones de diseño para conseguir la
solución al problema planteado. Durante la implementación se toma la solución conceptual y
se programa el software utilizando las herramientas adecuadas a la plataforma consiguiendo
una plataforma base que consta de controladores de hardware, sistema operativo y la apli-
1cación específica a la misión. Las pruebas se desarrollan sobre el sistema integrado en sus
tres subsistemas básicos mediante la puesta en funcionamiento del satélite por un tiempo
prolongado usando la técnica denominada hardware on the loop simulation adecuada para
probar sistemas embebidos complejos.



\section{Objetivos Generales}

El principal objetivo de este trabajo es diseñar e implementar una plataforma abierta que sirva como interfaz entre un satelites y sus payloads.  En este contexto, abierta significa que los usuarios tendrán acceso completo a las especificaciones de diseño y que pueden construir, modificar y extender los diseños a su propio gusto.  El hecho de definir una interfaz estándar entre el bus y los payloads, permite acelerar los tiempos de desarrollo para múltiples satélites así como también focalizar el desarrollo de payloads.

\section{Objetivos Específicos}

\section{Estructura}